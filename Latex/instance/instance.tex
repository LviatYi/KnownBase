% documentclass 控制序列
%   调用 article 文档类。
%   文档类是一些格式的集合。
%   部分控制序列还有被方括号 [] 包括的可选参数。
%-- \documentclass{article}
%   调用 ctexart 文档类。
%   设置文档类选项 UTF8
\documentclass[UTF8]{ctexart}


% ---------------------------------------- 导言区 ----------------------------------------

% 控制序列 documentclass 与 begin 之间的内容为 导言。
%   导言区的内容主要影响文档格式。

% -------------------- 加载宏包 --------------------

% usepackage 控制序列
%   加载 amsmath 宏包。
\usepackage{amsmath}
%   加载 anyfontsize 宏包。
%       允许连续缩放字体大小。
%       也可以使用 lmodern 字体解决字体大小相关警告。
\usepackage{anyfontsize}
%-- \usepackage{lmodern}
\usepackage{xfrac}
\usepackage{graphicx}
\usepackage{geometry}
\usepackage{fancyhdr}
\usepackage{setspace}

% -------------------- 定义文档属性 --------------------

% title 控制序列
%   定义 Title。
\title{Latex 实例}
% author 控制序列
%   定义 Author。
\author{LviatYi}
% date 控制序列
%   定义 Date
%   \today 插入编译时日期。
\date{\today}

% -------------------- 版面设置 --------------------

%   页边距
%   geometry.germetry
%       纸张大小
\geometry{papersize={210mm,297mm}}
%       页边距
\geometry{left=2cm,right=2cm,top=3cm,bottom=3cm}

%   fancyhdr.*
%   设置页眉或页脚。
%   设定 pagestyle。
\pagestyle{fancy}
%   EOLCRHF 选项用于指定位置。
%       E   偶数页显示
%       O   奇数页显示
%       L   左部
%       C   中部
%       R   右部
%       H   页眉部
%       F   页脚部
\fancyhf[LH]{LviatYi}
\fancyhf[CH]{\today}
\fancyhf[RH]{158********}
\fancyhf[CF]{\thepage}
%   设置页眉分割线宽度
\renewcommand{\headrulewidth}{1pt}
%   设置页脚分割线宽度
\renewcommand{\footrulewidth}{0pt}
\renewcommand{\headwidth}{\textwidth}

%   行间距
%   setspace.*
%   设置行距为字号的 1.5 倍。
\onehalfspacing

%   段间距
%   增加 0.4em 至段间距。
%   减少则为负。
\addtolength{\parskip}{.4em}

% ---------------------------------------- 正文环境 ----------------------------------------

% begin 控制序列
%   指定 document 环境。
%   与 end 控制序列成对出现。
\begin{document}

% 控制序列 begin 与 end 之间的内容为 环境。
% 环境中的内容主要影响文档内容。

% maketitle 控制序列
%   按照预定格式展示 title author date。
\maketitle

% tableofcontents 控制序列
%   依照文章结构生成一个目录。
\tableofcontents

% section 控制序列
%   生成一个章节。
%   用于控制文章结构。
%   定义于文档类 article/ctexart
%   类似作用还有:
%       \subsection{}       生成一个子章节
%       \subsubsection{}    生成一个子子章节
%       \paragraph{}        生成一个段落
%       \subparagraph{}     生成一个子段落
%       \chapter{}          生成一个章节    (定义于文档类 report / ctexrep)
%       \part{}             生成一个章节    (定义于文档类 book / ctexbook)
\section{文章结构}

\subsection{章节}

\subsubsection{中文}

你好,世界!

\subsubsection{英语}

Hello world!

\subsubsection{法语}

Bonjour le monde!

\subsection{段落}

\paragraph{Self introduction}

% 两个换行表示一个换行。
Here is Lviat Yi,

a dreamer who wanted to be a game producer.

\subparagraph{School}

NUIST

\subparagraph{Major}

Software Engineering

\section{方程}

\subsection{两种数学模式}

\subsubsection{行内公式}
能量可以转化成质量,质量可以湮灭成能量。爱因斯坦使用 $E=mc^2$ 描述质量与能量之间的当量关系。

\subsubsection{行间公式}
%   也可以使用 \begin{math}..\end{math}
%   不推荐使用 $$..$$
\[E=mc^2\]

\subsection{上下标}

%   默认只作用一个字符。
%   若要作用多个字符,需要使用 {}。

\[z = r\cdot e^{2\pi i}\]

\[I = I_1 + I_2\]

\subsection{根式与分式}

根式与分式 $\sqrt{x}$, $\frac12$.

\[ \sqrt{x} \]

\[ \frac{1}{2} \]

%   为了保持合适行高,行内公式与行间公式的输出效果有差异。
%   可以使用 \dfrac 强制行内模式的分式显示为行间模式的大小。
%   反之使用 \tfrac。

强制行间模式大小 $\dfrac12 $

强制行内模式大小
\[ \tfrac{1}{2} \]

%   xfrac.sfrac
微小行内分式 $\sfrac 1 2$

繁分式 \[\cfrac{1}{1+\cfrac1x}\]

\subsection{运算符}

小型且未被 Latex 使用的运算符可以直接输入。

\subsubsection{需要控制序列生成的运算符}

%   \; 用于生成一个空格
\[ \pm\; \times \; \div\; \cdot\; \cap\; \cup\;
    \geq\; \leq\; \neq\; \approx \; \equiv \]

%   连加 连乘 极限 积分
$ \sum_{i=1}^n i\quad \prod_{i=1}^n $

%   使用 \limits 和 \nolimits 显式指定是否压缩下标。
$ \sum\limits _{i=1}^n i\quad \prod\limits _{i=1}^n $

\[ \lim_{x\to0}x^2 \quad \int_a^b x^2 dx \]
\[ \lim\nolimits _{x\to0}x^2\quad \int\nolimits_a^b x^2 dx \]

%   多重积分
\[ \iint\quad \iiint\quad \iiiint\quad \idotsint \]

\subsection{定界符(括号)}

\[ \Biggl(\biggl(\Bigl(\bigl((x)\bigr)\Bigr)\biggr)\Biggr) \]
\[ \Biggl[\biggl[\Bigl[\bigl[[x]\bigr]\Bigr]\biggr]\Biggr] \]
\[ \Biggl \{\biggl \{\Bigl \{\bigl \{\{x\}\bigr \}\Bigr \}\biggr \}\Biggr\} \]
\[ \Biggl\langle\biggl\langle\Bigl\langle\bigl\langle\langle x
    \rangle\bigr\rangle\Bigr\rangle\biggr\rangle\Biggr\rangle \]
\[ \Biggl\lvert\biggl\lvert\Bigl\lvert\bigl\lvert\lvert x
    \rvert\bigr\rvert\Bigr\rvert\biggr\rvert\Biggr\rvert \]
\[ \Biggl\lVert\biggl\lVert\Bigl\lVert\bigl\lVert\lVert x
    \rVert\bigr\rVert\Bigr\rVert\biggr\rVert\Biggr\rVert \]

\subsection{省略号}

\[ x_1,x_2,\dots ,x_n\quad 1,2,\cdots ,n\quad
    \vdots\quad \ddots \]

\subsection{更多高阶技巧}

Latex 公式参考: https://www.zybuluo.com/codeep/note/163962

\section{图片与表格}

\subsection{图片}

%   graphicx.includegraphics
%   缩放图片的宽度至页面宽度的 90%,图片的总高度会按比例缩放。
\includegraphics[width = .9\textwidth]{assets/pic/Lviat Logo.png}

\subsection{表格}

%   指定 tabular 环境。
%   其提供简单的表格功能。
%   定义列格式,分别用 l、c、r 定义 居左、居中、居右
\begin{tabular}{|l|c|r|}
    %   \hline 命令表示横线。
    \hline
    %   使用 & 分列,使用 \\ 换行。
    操作系统   & 发行版   & 编辑器    \\
    \hline
    Windows    & MikTeX   & TexMakerX \\
    \hline
    Unix/Linux & teTeX    & Kile      \\
    \hline
    Mac OS     & MacTeX   & TeXShop   \\
    \hline
    通用       & TeX Live & VSCode    \\
    \hline
\end{tabular}

\section{浮动}

%   htbp 选项用于指定插图位置。
%       h   here        这里
%       t   top         页顶
%       b   bottom      页尾
%       p   float page  浮动页(专门放浮动体的单独页面或分栏)
\begin{figure}[htbp]
    %   定义居中。
    \centering
    \includegraphics[width = .5\textwidth]{assets/pic/Lviat Logo.png}
    %   设置插图标题。
    \caption{LviatYi Logo}
    \label{fig:myphoto}
\end{figure}

\section{版面设置}

\subsection{页边距}

见导言区。

\subsection{页眉与页脚}

见导言区。

\subsection{行间距与段间距}

见导言区。

\end{document}

% ---------------------------------------- 无效区 ----------------------------------------

% end 控制序列后的内容是无效的。
% 无效的
Hello, world!



